\documentclass[12pt]{article} % Anonymized submission
\usepackage{fullpage}
% Packages
\usepackage{amsfonts}
\usepackage{amsmath}
%\usepackage{amsthm}
\usepackage{graphicx}
\usepackage{times}

\title{Designing online learning algorithms using Potentials}
\author{Yoav Freund}

%\theoremstyle{plain}
\usepackage{amsmath,bm}

\newtheorem{lemma}{Lemma}
\newtheorem{claim}[lemma]{Claim}
\newtheorem{theorem}[lemma]{Theorem}
\newtheorem{corollary}[lemma]{Corollary}
\newtheorem{definition}{Definition}
\newtheorem{question}{Question}

\newcommand{\note}[1]{\par{\bf Note:}#1\par}
\newcommand{\notem}[1]{{\marginpar{\tiny #1}}}

\newcommand{\figline}{\rule{\textwidth}{1pt}}

%\newcommand{\proof}{\noindent{\bf Proof:} }
%\newcommand{\qed}{\rule{0.7em}{0.7em}}

\newcommand{\newmcommand}[2]{\newcommand{#1}{{\ifmmode {#2}\else\mbox{${#2}$}\fi}}}
\newcommand{\newmcommandi}[2]{\newcommand{#1}[1]{{\ifmmode {#2}\else\mbox{${#2}$}\fi}}}
\newcommand{\newmcommandii}[2]{\newcommand{#1}[2]{{\ifmmode {#2}\else\mbox{${#2}$}\fi}}}
\newcommand{\newmcommandiii}[2]{\newcommand{#1}[3]{{\ifmmode {#2}\else\mbox{${#2}$}\fi}}}

\newcommand{\E}[2]{{\bf E}_{#1}\left[ #2 \right]}
%\newcommand{\D}[2]{{\bf D}_{#1}\left[ #2 \right]}
\newcommand{\D}{{\cal D}}

\renewcommand{\P}[2]{{\bf P}_{#1}\left[ #2 \right]}
\newcommand{\reals}{\mathbb{R}}

\newcommand{\argmin}{\mbox{argmin}}
\newcommand{\argmax}{\mbox{argmax}}
\newcommand{\SP}[1]{{\cal P}^{#1}} % strictly positive function of
                                   % order i
\newcommand{\R}{R}      % Cumulative regret
\newcommand{\vR}{\mathbf{R}} %regret vector

\renewcommand{\r}{r}      % Instantaneuous regret
\newcommand{\state}{{\bf \Psi}}
\newcommand{\1}[1]{{\mathbf 1}\left[#1\right]} % point mass at #1

\newcommand{\pot}{\phi}
\newcommand{\potPQ}{\pot_{\learnerM,\adversM}}
\newcommand{\potLA}{\pot_{\l,\a}}
\newcommand{\finalPot}[1]{\pmb{\phi}_{#1}}
\newcommand{\finalPotT}{\finalPot{T}}
\newcommand{\finalPotR}{\finalPot{\realT}}
\newcommand{\upperpot}{\pot_{\learnerM}^{\downarrow}}
\newcommand{\upperpotb}{\pot_{\learnerMb}^{\downarrow}}
\newcommand{\upperpotd}{\pot_{\learnerMd}^{\downarrow}}
%\newcommand{\upperpotj}{\pot_{\learnerM(j)}^{\downarrow}}
\newcommand{\upperpotMdk}{\pot_{\learnerMdk}^{\downarrow}}
\newcommand{\upperpotMdj}{\pot_{\learnerMdj}^{\downarrow}}


\newcommand{\lowerpot}{\pot_{\adversM}^{\uparrow}}
\newcommand{\lowerpotb}{\pot_{\adversMb}^{\uparrow}}
\newcommand{\lowerpotd}{\pot_{\adversMd}^{\uparrow}}
\newcommand{\lowerpotj}{\pot_{\adversM(j)}^{\uparrow}}
\newcommand{\lowerpotMdk}{\pot_{\adversMdk}^{\uparrow}}
\newcommand{\lowerpoteven}{\lowerpot_{\evensplit}}

\newcommand{\realT}{\mathcal{T}}  % the final time for the real-time
% game
\newcommand{\Tset}[1]{\pmb{T}_{#1}}
\newcommand{\Ilat}[1]{\pmb{I}_{#1}}
\newcommand{\Klat}[1]{\pmb{K}_{#1}}
\newcommand{\score}{\Phi}
\newcommand{\upperscore}[1]{\score_{#1}^{\downarrow}}
\newcommand{\lowerscore}[1]{\score_{#1}^{\uparrow}}

\newcommand{\upperscoreM}{\upperscore{\learnerM}}
\newcommand{\upperscoreMd}{\upperscore{\learnerMd}}
\newcommand{\upperscoreMdk}{\upperscore{\learnerMdk}}

\newcommand{\lowerscoreM}{\lowerscore{\adversM}}
\newcommand{\lowerscoreMd}{\lowerscore{\adversMd}}
\newcommand{\learnerM}{P}
\newcommand{\learnerMb}{P_I}
\newcommand{\learnerMd}{P_D}
\renewcommand{\l}{\learnerM}
\newcommand{\learnerMdk}{P_{D(k)}}
\newcommand{\learnerMdj}{P_{D(j)}}


\newcommand{\legalLearner}{{\cal L}}

\newcommand{\adversM}{Q}
\newcommand{\adversMb}{Q_I}  %Integer time game
\newcommand{\adversMd}{Q_D}  % Discrete time game
\newcommand{\adversMdk}{Q_{D(k)}}  %Discrete time game with step of
                                %size 2^{-k}


% \renewcommand{\a}{\adversM}
%\newcommand{\legalAdversary}{{\cal A}}
\newcommand{\agloss}{v}
\newcommand{\Bias}{B}
\newcommand{\deltat}{\Delta t}

\newcommand{\diffweight}{\l^d}   % learner strategy based on taking a difference
\newcommand{\upperpotdiff}{\upperpot_{\diffweight}}


\newcommand{\Binom}{\mathbb{B}}
\newcommand{\radsum}{\Binom(s_1,\ldots,s_T)}
\newcommand{\var}{\mbox{Var}}
\newcommand{\V}{V}


\newcommand{\at}[1]{\left\{ \left. #1
    \right|_{\begin{tiny}\begin{matrix}
          \tau,\rho=\\t_i,\R \end{matrix} \end{tiny}}
    \pot(\tau,\rho)\right\}}
\newcommand{\att}[1]{\left\{ \left. #1  
\right|_{\begin{tiny}\begin{matrix} \tau,\rho=\\t_i+g \deltat_i,\R_i+g
      r_i \end{matrix} \end{tiny}}
\pot(\tau, \rho)\right\}}


\newcommand{\atI}[1]{\left\{ \left. #1  
\right|_{\begin{tiny}\begin{matrix}
      x,y=\\x_0,y_0 \end{matrix} \end{tiny}}
f(x,y) \right\}}
\newcommand{\atII}[1]{\left\{ \left. #1
\right|_{\begin{tiny}\begin{matrix}
      x,y=\\x_0+t\Dx,y_0+t\Dy \end{matrix} \end{tiny}}
f(x,y) \right\}}


\newmcommandi{\paren}{\left({#1}\right)}
\newmcommandi{\brac}{\left[{#1}\right]}




\begin{document}


\maketitle
\begin{abstract}
  Abstractly speaking.
\end{abstract}

\section{Introduction}

Online prediction with expert advise has been studied extensively over
the years and the number of publications in the area is vast (see
e.g.~\cite{vovk1990aggregating, feder1992universal,
  littlestone1994weighted, cesa1997use, cesa2006prediction}.

Here we focus on a simple variant of online prediction with expert
advice called {\em the decision-theoretic online learning game}
(DTOL)~\cite{freund1997decision}, we  consider the {\em
  signed} version of the online game.

DTOL is a repeated zero sum game between a {\em learner} and an {\em
  adversary}. The adversary controls the losses of $N$ experts, or
actions, while the learner controls a distribution over the actions.
~\\
Iteration $t=1,\ldots,T$ of the game consists of the following steps:
\begin{enumerate}
    \item The learner chooses a distribution $P_j^t$ over the
      actions $j \in \{1,\ldots,N\}$. 
    \item The adversary chooses an {\em instantaneous loss} for each
      of the $N$ actions: \\
      $l_j^t \in [-1,+1]$ for $j \in \{1,\ldots,N\}$.
    \item The learner incurs an {\em instantanous expected loss} defined as
      $\ell^t = \sum_{j=1}^N P_j^t l_j^t$
\end{enumerate}

Using standard notation we denote the {\em cumulative loss} of
action $j$ for times $1,\ldots,T$ by $L^T_j = \sum_{t=1}^T l_j^t$.
Similarly, we denote the {\em cumulative loss} of the learner by
$L_\ell^T = \sum_{t=1}^T \ell^T$.
%The {\em instantaneous regret} of
%the learner with respect to action  $j$ is defined as
%$r_j^t = \ell^t-l_j^t$.
The {\em cumulative regret} of the learner
with respect to action $j$ is $\R_j^T = \sum_{t=1}^T r_j^t =
L_\ell^T -L_j^T $. Finally, for any $0\leq \epsilon \leq 1$ we define the $\epsilon$ regret
to $R^T_\epsilon$ to be value $v$ such that the
fraction of values of $\R_j^T$ that are smaller than $v$ is at most
$\epsilon N$.

The goal of the learner is to minimize the maximal regret at time $T$
$R_{\epsilon}^T \doteq \max_j \R_j^T = L_\ell^T - \min_j L_j^T$.
The goal of the adversary is to maximize $R_{\epsilon}^T$.

Our goal is to identify algorithms that improve over known bounds of the
form $O(\sqrt{T \ln \frac{1}{\epsilon}})$.
%{\bf Needs comparison to existing results, NormalHedge, Squint, square
% loss bounds}

We are interested in bounds the hold simultanously for all values
of $\epsilon$. Denote the distribution over regrets at some fixed
iteration by $\mu$. We say that the distribution $\mu$ satisfies the regret function $B$ if

\begin{definition}[Simultanous regret
  bound] \label{def:unif-regret-bound}
  Let $B: \reals \to [0,1]$ be a non-increasing function which maps
  regret bounds to probabilities.
  
  A distribution over regrets $\mu$ is simultanously bounded by $B$ if
  \[
    \forall r \in \reals \;\; \P{\R \sim \mu}{\R \geq r} \leq B(r)
  \]
Where $B: \reals \to [0,1]$ is a non-increasing function.
\end{definition}

Our main tool for designing algorithms and proving bounds is potential
functions.

We say that  A distribution over regrets $\mu$ satisfies
the potential function $\pot$ if 
%We say that the bound $B_1$ dominates $B_2$ if $\forall r  B_1(r) \leq
%B_2(r)$ and $\exists r B_1(r) <B_2(r)$

\begin{definition}[Average potential bound] \label{def:aver-potential-bound}
  A distribution over he reals $\mu$ satisfies the average
  potential function $\pot$ if
  $$\E{\R \sim \mu}{\pot(\R)} \leq 1$$
  Where $\pot: \reals \to \reals^+$ is a non decreasing function. 
\end{definition}

\begin{theorem}
 A distribution $\mu$ is simultanously bounded by $B$ if and only
 if it satisfies the average potential bound with $\phi(\R) = B(\R)^{-1}$
\end{theorem}

Unlike uniform regret bounds, that are rather opaque, one can easily
write a single step backward recursion for potentials. In other words,
given a potential function at iteration $t$, one can construct
potential functions for iteration $t-1$. This povides a way 
for optimizing the potential function.

The original DTOL game does not seem to have min-max
strategies. However, in an extended version of the game we can
characterize min-max strategies. The extended game provides the
adversary with additional possibilities, but not the learner. In other
words, the standard game is more restrictive to the adversary, which
means that upper bounds on the regret for the extended game also hold
for the restricted game.

\begin{itemize}
\item min-max for known $T$
\item Almost min/max for unknown $T$
\item Simultanously for all $\epsilon$
\item Replacing $T$ with $V_T$
\item Interpretation of the limit game.
\end{itemize}

\paragraph*{Potential Based Algorithms}
Many of the known algorithms for online learning are based on a {\em
  potential function}. Roughly speaking, a potential function
$\pot(r,t)$ associates an action whose cumulative regret $r$ on iteration
$t$ a non negative potential. The weight that the learner associates
with the action is (approximately) the gradient of the potential
function.

The standard technique for proving regret bounds is to show an upper
bound on the average potential which can then be translated into an
upper bound on the regret relative to the best $\epsilon$ percentile.

While potential-based online learning algorithms have a long history,
other approaches such as {\em follow the leader} and it's variants
provide similar bounds and are often much more efficient.

Surprisingly, we prove that there is a one-to-one correspondence
between a simultanuous (deterministic) regret guarantees and a
potential function that corresponds to the reciprocal of the bound. We
thus focus on potential based learning algorithms.

Given a potential function for iteration $t$ we can naturally
calculate potential functions for iterations $t-1,t-2,\ldots$. Using
this backwards recursion we can solve games with a known
horizon. 


\section{Extending the game}

Our goal is to describe a potential-based algorithm which is min-max
optimal. To achieve this we give the adversary more options than it
has in the standard game.

Specifically, we arbitrary division and choosing the step size.


\section{The algorithm}

We now describe the learning algorithm for the extended game.

We define the tail probability of a normal distribution with mean zero
and variance $t$ to be
\begin{equation}
\epsilon(R,t) = \sqrt{2 \pi t} \int_R^\infty e^{-\frac{\rho^2}{2t}} d\rho
\end{equation}

As the tail probability is monotonically decreasing from 1 to 0, the
inverse function $R(\epsilon,t)$ is well defined so that
$\epsilon(R(\epsilon_0,t),t)=\epsilon_0$ and $R(\epsilon(R_0,t),t) =R_0$

We define a potential that is the reciprocal of the tail:
$$\pot(R,t) = \frac{1}{\epsilon(R,t)}$$

On iteration $t$ our algorithm assigns to expert $j$, whose
regret on iteration $t$ is $R^t_j$ the weight
$$w^t_j = \pot(R_j+2,t+1) - \pot(R_j-2,t+1)$$
And the normalized weights are $P_j^t = w^t_j/Z_t$ where $Z_t =
\sum_{j=1}^N w_j^t$

Our main result is
\begin{theorem}
  There exists a parameter-free DTOL algorithm and a constant $c>0$
  such that for any $N$ loss sequences of length $t$, for any $1>\epsilon>0$ we
  have that
  \[
    \#\left\{ 1 \leq i \leq N \left| R_i \geq R(\epsilon,t+c \log t)
    \right.\right\} \leq \epsilon N
  \]
\end{theorem}

\section{Discrete time game}
\label{sec:discrete}

Our analysis is based on defining an expanded version of DTOL game which
allows for a tighter analysis. The extended game provides more options
to the adversary, but not to the learner. In particular, it allows the
adversary to play the standard DTOL game. As a consequence, regret bounds
that hold for the extended game also hold for standard DTOL in which
the adversary is more restricted.

In the extended game we game we use separate between the notion of
iteration $i=1,2,3,\ldots$ and the notion of time $0=t_0 \leq t_1 \leq
t_2 \leq \cdots$. The gap $t_{i+1}-t_i \leq 1$ and therefor $t_i\leq
i$. Our regret bounds will involve $t_i$ instead of $i$ and will
be at most as large as the bounds on standard DTOL. 

The first extension of DTOL is to allow the adversary to choose the
range of the losses at each iteration. In standard DTOL this range is fixed to
$[-1,+1]$. In the extended game we allow the adversary to choose the
range $[-s_i,+s_i]$ for some $s_i \in [0,1]$. It is not hard to see
that if this was the only change then the adversary would always
choose $s_i=1$. To make the choice $s_i<1$ viable


To make it worthwhile for the adversary to choose $s_i<1$
we replace the ``number of iterations'' in our regret bounds with a
notion of ``time'' that is defined as follows. Let $t_i$ be a real
number that defines the time at iteration $i$, $t_0=0$ and
$t_{i+1} = t_i + s_i^2$
we assume there exists a finite $n$ such that $t_n = T$.

We need to put a bound on $\Bias(t_i,R)$ because without a bound the
bias can be $s_i$ and the cumulative  bias after unit time would be
$s_i/s_i^2 = 1/s_i$ which goes to infinity as $s_i \to 0$. Infinite
biases make that total loss undefined. 

We will later give some particular potential functions for which no
a-priori knowledge of the termination condition is needed. The
associated bounds will hold for any iteration of the game.
\vspace{1cm}

On iteration $i=1,2,\ldots$
\begin{enumerate}
\item  If $t_{i-1}=T$ the game terminates.
\item The adversary chooses a {\em step size} $0<s_i\leq 1$, which advances
  time by $t_i = t_{i-1}+s_i^2$.
\item Given $s_i$, the learner chooses a distribution $\learnerM(i)$ over $\reals$.
\item The adversary chooses a mapping from $\reals$ to distributions
  over $[-s_i,+s_i]$: $\adversM:\reals^2 \to \Delta^{[-s_i,+s_i]}$ \newline
  such that $\Bias(t_i,\R) \doteq \E{y \sim \adversM(t_i,\R)}{y} \leq
  c s_i^2$
  for all $\R \in \reals$.
\item The aggregate loss is calculated:
  \begin{equation} \label{eqn:ell-discrete}
    \ell(t_i)=\E{\R \sim \state(t_i)}{\learnerM(t_i,\R)
      \Bias(t_i,\R)}
  \end{equation}
\item The state is updated. The expectation below is over
  distributions. and the notation $G \oplus \R$ means
  that distribution $G$ over the reals is shifted by the amount
  defined by the scalar $\R$:
  $$\state(t_i) = \E{\R \sim \state(t_{i-1})}{\adversM(t_i)(\R)\oplus (\R-\ell(t_i))}
  $$
\end{enumerate}

When $t_i=T$ the game is terminated, and the final value is calculated:
$$\score(T) =\E{\R \sim \state(T)}{\finalPot}$$


\section{Equivalence between simultanous regret bounds and potentials}

A standard technique in the analysis online learning is to prove an
upper bound on the average potential and use that upper bound to prove
an upper bound on the regret. Here we show a sense in which the
converse is also true: if we have a {\em unifom} upper bound on the
regret, then there is an upper bound on the average
potential. 

We fix the time step and consider an arbitrary probability measure $\mu$ over
the real line that defines the regret distribution. We give two
characterizations for $\mu$, one bounding the regret, the other
bounding the potential. We then show a one to one relationship between
the two characterizations.

\begin{definition}[Simultanous regret bound] \label{def:unif-regret-bound}
  A distribution over the reals $\mu$ is simultanously bounded by $B$ if
  \[
    \forall a \in \reals \;\; \P{\R \sim \mu}{\R \geq a} \leq B(a)
  \]
Where $B: \reals \to [0,1]$ is a non-increasing function.
\end{definition}

Next we define the average potential bound.
\begin{definition}[Average potential bound] \label{def:aver-potential-bound}
  A distribution over he reals $\mu$ satisfies the average
  potential function $\pot$ if
  $$\E{\R \sim \mu}{\pot(\R)} \leq 1$$
  Where $\pot: \reals \to \reals^+$ is a non decreasing function. 
\end{definition}

We now show a one-to-one relationship between these two types of bounds

\begin{theorem}
 A distribution $\mu$ is simultanously bounded by $B$ if and only
 if it satisfies the average potential bound with $\phi(\R) = B(\R)^{-1}$
\end{theorem}
\begin{proof}
  \begin{itemize}
  \item
  {\bf $\mu$ satisfies a simultanous bound for $B$ if it satisfies an
  average potential bound for $\pot = B^{-1}$}\\
Assume by contradiction that $\mu$ does not satisfy the simultanous bound. In
other words there exists $a \in \reals$ such that
$\P{\R \sim \mu}{\R > a} > B(a)$. From Markov inequality and the fact
that $\phi$ is non decreasing we get
\[
  \E{\R \sim \mu}{\pot(\R)} \geq \phi(a) \P{\R \sim \mu}{\R > a} >
  \phi(a) B(a) = \frac{B(a)}{B(a)}=1
\]
but $ \E{\R \sim \mu}{\pot(\R)} >1$ contradicts the average potential
assumption for the potential $\phi(\R) = B(\R)^{-1}$
\item
{\bf $\mu$ satisfies an
  average potential bound for $\pot = B^{-1}$ if it satisfies a simultanous bound for $B$}\\
As $\phi$ is a non-decreasing function, and assuming $\R,\R'$ are drawn
independently at random according to $\mu$:
\begin{eqnarray}
  \E{\R \sim \mu}{\pot(\R)} & = & \E{\R \sim \mu}{\pot(\R)
                                  \P{\R' \sim \mu}{\phi(R') \geq \phi(\R)}} \\
                            & \leq & \E{\R \sim \mu}{\pot(\R)
                                     \P{\R' \sim \mu}{R' \geq \R}} \\
                            & < & \E{\R \sim \mu}{\pot(\R) B(\R)} \\
                            & = & \E{\R \sim \mu}{\frac{B(\R)}{B(\R)}}
                                  = \E{\R \sim \mu}{1} = 1
\end{eqnarray}
\end{itemize}
\end{proof}

\section{Multiply convex potential functions}

\section{Backwards recursion of potentials functions}




\subsection{the rest}

We can now state the main claim of this paper.


Online learning can be described as and iterative game~\cite{}

We define the {\em state} of the online prediction game as

We define a potential function as ...

The tail bound can be written as a potential function...

We can work our way backwards.

A common approach to designing online learning algorithms is to define
a {\em potential function}. The potential function
$\pot:\reals \to \reals$ is a
positive, continuous and non-decreasing function of the regret.
One popular potential function is the exponential function
which has the form $\pot(\R) = \exp(\eta \R)$ where $\eta>0$
is the {\em learning rate}. See~\cite{cesa2006prediction} for an
extensive review of the use of potential functions for online learning.

A central quantity in the design and analysis of potential based
algorithms is the {\em average potential} which we refer to here as the {\em
  score}. The score at time $k$ is defined as:
\begin{equation} \label{eqn:score}
  \score^k = \frac{1}{N} \sum_{j=1}^N \pot(R_j^k)
\end{equation}

The fact that $\pot(\R)$ is positive and non-decreasing implies an upper
bound on the regret w.r.t. any expert.
Suppose that the score at time $t$ is upper bounded by $A$ and that the learner
suffers regret $B$ with respect to at least one expert:
$B=\max_j(\R_j^T)$, then  $\pot(B) \leq N \score_T \leq NA$.

In most formulations of this technique, the potential function $\pot$ is
fixed a-priori, and the learner's strategy is designed based on this function.
This raises a natural question:
\begin{question} \label{quest:optimality}
  Is there an optimal potential function $\pot$ for DTOL?
\end{question}

Without further constraints, this question is ill-defined. We take
several steps to better define the question.
\begin{enumerate}
\item We allow the potential function to depend on the
  iteration number $i$, i.e. we study potential functions of the form
  $\pot(i,\R)$. ~\footnote{Potentials that depend on time were used
    for NormalHedge and related
    algorithms~\cite{chaudhuri2009parameter,luo2015achieving}}
\item We fix the number of iterations $T$. This assumption will later be removed.
\item We fix the final potential $\pot(T,\cdot)$. This function
  is required to have defined and
  strictly positive derivatives of degrees zero to four.\footnote{The zeroth derivative
    is the function itself. Therefor positivity of derivatives 0 and
    1 implies the standard assumption
  that $\pot(T,\R)$ is positive and increasing.} This  
assumption is satisfied by most commonly used potential functions,
including the exponential potential, the
NormalHedge potential and polynomial potentials of degree higher than four.
\end{enumerate}
This leads to a more specific question:
\begin{question} \label{ques:goodStrategies}
  Given the length of the game $T$ and the final potential function
  $\pot(T,\R)$, can we define the best potentials $\pot(i,\R)$ for
  $0 \leq i < T$
\end{question}

We give a qualified answer to this question. Rather than finding a {\em
  single} potential function, we associate an {\em upper
  potential} with learner strategy $\l$ and a {\em lower potential}
with adversarial strategy $\a$.

We start by fixing both $\a$ and $\l$. As we show in Definition~\ref{def:back-recursion} and
Lemma~\ref{lemma:constant-score}, for any pair of strategies 
$\a$ and $\l$  and for any iteration $0\leq i \leq T$ there exists a potential function
$\pot_{\a,\l}(i,\cdot)$ such that if we define the score
$\score_{\a,\l}(i,\vR^{i})$ to be 
\begin{equation} \label{eqn:potential}
  \score_{\a,\l}(i,\vR^i) = \frac{1}{N} \sum_{j=1}^N \pot_{\a,\l} \left(T-1,\R_j^{i} \right)
\end{equation}

then $\score_{\a,\l}(0,\vR^0)=\cdots = \score_{\a,\l}(T,\vR^T)$.

Using Equation~\ref{eqn:potential} we define an {\em upper
  potential} and a {\em lower potential} for each iteration $0 \leq i
\leq T$
\begin{equation}
  \upperpot_\l(i,\R^i) = \max_\a \pot_{\a,\l}(i,\R^i) \mbox{ and }
  \lowerpot_\a(i,\R^i) = \min_\l \pot_{\a,\l}(i,\R^i)
\end{equation}
Which are given given inductive definitions in Equations~(\ref{eqn:recursive-upper},\ref{eqn:recursive-lower})

Clearly, for any $\a,\l, i, \R$ we have that $\lowerpot_\l(i,\R) \leq
\upperpot_\a(i,\R)$. Our goal is to find a pair of strategies $\a,\l$ such
that  $\lowerpot_\l(i,\R) = \upperpot_\a(i,\R)$. That would imply that
$\a$ and $\l$ are min-max optimal strategies.

It is unclear whether min-max strategies exist for standard DTOL.
In order to close the apparent gap between the upper and lower
potentials and identify the min-max strategies we {\em expand} the game.
Here expansion means giving the {\em  adversary} more choices
while keeping the learner's choices unchanged. As a consequence, any
upper bounds guarantees that hold for the expanded game also hold for
the standard DTOL (but the lower bounds might not). 

We make two expansion steps. In the
first, we allow the adversary to arbitrarily divide experts. In the
second we allow the adversary to choose the range of allowed
instantanous losses. Using both expansions  we identify the min/max strategies
and the min/max potential function thereby answering question
\ref{ques:goodStrategies}.

The first expansion gives the adversary the ability to arbitrarily split
experts. Intuitively, rather than associating a single loss with each
expert, The adversary can arbitrarily divide the the expert into
sub-experts and assign a different loss to each. This is similar to~\cite{chernov2010prediction}
where the number of experts is not known in advance
% To understand the significance of the ability to split experts,
% consider the well studied ``follow the leader'' methods. suppose that
% at each iteration the leader splits into into two equal-weight parts,
% one part suffer loss $1$ and the other loss $-1$. Such a leader is
% hard to follow.

We define the {\em even split} adversarial strategy $\evensplit$ as
one which, at each iteration splits {\em each current expert} into two equal parts,
one incurs loss 1 and the other loss -1. The result is that at
iteration $n$ we have $2^n$ experts, each of weight $2^{-n}$, each
corresponding to a loss sequence in $\{-1,+1\}^n$. The cumulative loss
of the experts has a binomial distribution
$\Binom(n,1)$~\footnote{\label{def:binomial} We use
  $\Binom(n,s)$ to denote the binomial distribution that assigns
  probability ${n \choose j} 2^{-n}$ to the point $(n-2j)s$ for
  $j=0,\ldots,n$}. The symmetry of $\evensplit$ implies that the loss
of the learner against this adversary is zero independently of the
learner choices. Therefor the cumulative regret is also Binomially
distributed according to $\Binom(n,1)$.  The final score is equal to the
expected value of the final potential
$\E{\R \sim \Binom(T,1)}{\pot(T,\R)}$. 
As we will show there is a simple strategy
for the learner that guarantees an upper bound of $\E{\R \sim \Binom(T,2)}{\pot(T,\R)}$. 
The upper and lower bounds are both based on binomial
distribution. However, as the step sizes are $1 <2$, these
strategies are not a min/max pair.

Intuitively, in order to achieve min/max optimality we need to make
the step size of the adversary equal the step size of the learner. We
next explain how the step size for the learner is reduced to the step
size of the adversary. 

We achieve that by taking the second expansion step.
In this expansion we allow the the adversary to set the step ranage in
iteration $i$ to $[-s_i,+s_i]$ for any $0 \leq s_i \leq 1$. In
Section~\ref{sec:discrete} we give a full description of the game.
For now, suppose that the adversary chooses $s_i=1/m$ for some large natural
number $m$ and for all $i$. Scaling $\evensplit$ to be $\pm 1/m$ with
equal probabilities we find that the lower score is equal to
$\E{\R \sim \Binom(T,1/m)}{\pot(T,\R)}$. It is not hard to see that if
$m to \infty$ while $T$ remains constant then the binomial
distribution degenerates to the delta function. In order to avoid this
degeneracy we set $T=m^2$. The binomial distribution
$\Binom(m^2,1/m)$ has variance one for all $m$ and converges to the
standard normal as $m \to \infty$. In Section~\ref{sec:discrete} we
show that in this limit the difference between the upper and lower
potentials converges to zero.

To show that this limit is the min/max, we need to show that, in fact,
the optimal choice for the adversary is to set $m$ arbitrarily
high. If the adversary prefers a finite $m$, then this is not the
min/max. This is where the strictly positive fourth derivative is
important. In Section~\ref{} we show that, if the final potential
function is in $\SP{4}$ then the lower potential strictly increases
with $m$.  Establishing the min/max properties of the limit
$m \to \infty$.

One benefit of working with the weiner process is that the backward
equations that define the potential transform, in the limit, into the
Kolmogorov Backwards equations (KBE)
The potential function $\lowerpot(t,\R) = \upperpot(t,\R) = \pot(t,\R)$
is the solution of the Kolmogorov backwards equation (KBE) with the
boundary condition defined by $\pot(T,\R)$. We have thus
answered Question~\ref{ques:goodStrategies} in the affirmative.

\newcommand{\parvec}{\vec{\theta}}
Having identified the min/max strategies for a given final potential
function $\pot(T,\R)$ we next turn to choosing $\pot(T,\R)$. We say
that a parametrized potential function $\pot(\parvec_T,\R)$ 
is {\em compatible} with KBE if the solution to KBE with boundary condition
$\pot(T,\R) = \pot(\parvec_T,\R)$ can be written in the form
$\pot(t,\R) = \pot(\parvec_t,\R)$. In
Section~\ref{sec:self-consistent} we identify two KBE compatible
functions, one corresponds to the exponential potential, the other to
the NormalHedge potential. An additional benefit of characterizing the
solution as a solution of KBE is that the solution can be extended to
$t>T$. This removes the need to know $T$ a-priori and yields an {\em
  any time} algorithm.

Finally, we come back to Question~\ref{quest:optimality}. Intuitively,
we want $\pot(T,\R)$ (properly normalized) to increase as fast as
possible, without causiing the score to increase with $t$.  As we have
identified the worst case adversary to be Brownian motion, we can ask
what is the fastest increasing potential for which Brownian motion
will not increase the score. In section~\ref{sec:NormalHedge} we show
that an appropriate limit of NormalHedge has this distinction.

\section{Preliminaries} \label{sec:preliminaries}
The integer time game takes place on the set $(i,\R) \in \{0,1,\ldots,T\} \times \reals$,
where $i$ corresponds to the iteration number ,
and $\R$ corresponds to the (cumulative) regret.

As in the standard DTOL setting, an expert corresponds to a sequence
of cumulative regrets. However, unlike the standard DTOL the number
of experts is allowed to be infinite. In DTOL the {\em state}
of the learning process is defined by the by the regret of each of the
$N$ experts: $\langle \R_1(i),\ldots,\R_N(i) \rangle$. In order to
represent the regret of a potentially uncountable set of experts we
define the state as a distribution over possible regret values.  Thus
the {\em state} of the game on iteration $i$ is a distribution
(probability measure) over the real line, denoted $\state(i)$.  We
denote by $\R \sim \state(i)$ a random regret $\R$ that is chosen
according to the distribution corresponding to the state at iteration
$i$. Given a measureable function $ f:\reals \to \reals$ we define
$\D{\R\sim \state(i)}{f(\R)}$ to be the distribution of $f(\R)$ when
$\R \sim \state(i)$. We similarly define the expectd value of $f(\R)$
by $\E{\R\sim \state(i)}{f(\R)}$, and the probability of an event
${\mathbf e}$
defined using $f(\R)$ by $\P{\R\sim \state(i)}{{\mathbf e}(f(\R))}$.

The initial state $\state(0)$ is a point mass at $\R=0$. 
The state $\state(t)$ is defined by $\state(t_{i-1})$ and the choices made
by the two players as described in the next section.

The {\em final potential function} $\pot(T,\cdot)$ is predefined and
known to both sides. The final {\em score} is defined to be
\begin{equation} \label{eqn:FinalExpectedValue}
  \score(T) =\E{\R \sim \state(T)}{\finalPot}
\end{equation}
The goal of the learner is to minimize $\score(T)$ and the goal of
the adversary is to maximize it.

We assume that the final poptential is {\em strictly positive of degree $k$}, which is defined as follows:
\begin{definition}[Strict Positivity of degree $k$]
A function $f:\reals \to \reals$ is strictly positive of degree $k$, 
denoted $f \in \SP{k}$ if the derivatives of orders 0 to $k$:  
$f(x), \frac{d}{dx}f(x), \ldots, \frac{d^k}{dx^k}f(x)$ exist and are strictly positive.
\end{definition}

A simple lemma connects an upper bound on any score function in $\SP{1}$ with a bound on the regret.
\begin{lemma}
Let $\finalPot \in \SP{1}$, $\score(T) \leq U$ and $\epsilon=\P{\R \sim
  \state{t}}{\R >\R'}$ be the probability of the set of experts with
respect to which the regret is larger than $\R'$ Then

then the regret of the algorithm relative 
to the top $\epsilon$ of the experts is upper bounded by
\[
  \finalPot \leq U/\epsilon
\]
\end{lemma}

\section{Integer time game} \label{sec:integer}
We start with a setup in which time and iteration number are the same, i.e.
$t_i=i$. In this section we suppress $t_i$ and instead use the
iteration number $i=0,1,2,\ldots,T$.

We define the {\em state of the game} on iteration $i$: $\state(i)$ as
the distribution of regret over the experts.
Note that experts is allowed to be be un-countably infinite. In particular the
adversary can assign to the experts with regret $x$ at iteration $t$
an arbitrary distribution of losses in the range $[-1,+1]$.

The game is defined by three parameters:
\begin{itemize}
\item $T$ : The number of iterations
\item $\state(0) = \delta(0)$ is the initial state of the game which
  is a point mass distribution at 0. 
\item $\finalPot$ : The function that is in $\SP{2}$.
\end{itemize}

The transition from $\state(i)$ to $\state(i+1)$ is  defined by the
choices made by the adversary and the learner.

\begin{enumerate}
\item The learner chooses weights. Formally, $\learnerM(i,\cdot)$ is 
  a density over $\reals$:
\item The adversary chooses the losses of the experts. Formally 
  this is a mapping from $\reals$ to distributions
  over $[-1,+1]$: $\adversM(i): \reals \to \Delta^{[-1,+1]}$. We use
  $l \sim \adversM(i,\R)$ to denote the distribution over the
  instantaneous loss associated with iteration $i$ and regret $\R$.
\item The aggregate loss (also called ``the loss of the master") is
  calculated:
  \begin{equation} \label{eqn:agg-loss-complex}
  \ell_\l(i)=\E{\R \sim \state(i)}{\learnerM(i,\R) \E{l \sim \adversM(i,\R)}{l}}
  \end{equation}
  
  We define the {\em bias} at $(i,\R)$ to be  $\Bias(i,\R) \doteq \E{l
    \sim \adversM(i,\R)}{l}$ which allows us to rewrite
  Eqn~(\ref{eqn:agg-loss-complex}) as
  \begin{equation} \label{eqn:aggregate-loss}
    \ell_\l(i)=\E{\R \sim \state(i)}{\learnerM(i,\R) \Bias(i,\R)}
  \end{equation}
  note that $\Bias(i,\R)$ is in $[-1,1]$ and that $\ell_\l(i)$ is the mean
  of $\learnerM(i,\R) \Bias(i,\cdot)$. Note also that $-2 \leq y-\ell_\l(i) \leq 2$
  corresponds to the instantaneous regret.
  
\item The state is updated. 
  \begin{equation} \label{eqn:state-update}
    \state(i+1) = \D{\R \sim \state(i), l \sim \adversM(i,\R)}{\R + l -\ell_\l(i)}
  \end{equation}
  Which is the distribution of $\R+l -\ell_\l(i)$ where $\R$ is the
  cumulative regret at iteration $i$, whose distribution is to
  $\state(i)$, $l$ is the instantanous loss chosen according to the
  adversarial distribution $\adversM(i,\R)$ and $\ell_\l(i)$ is the
  average loss as defined above. 
\end{enumerate}

The final score is the mean of the potential according to the final
state, as given in Equation~(\ref{eqn:FinalExpectedValue}).
The goal of the learner is to minimize the final score and the goal of
the adversary is to maximize it.
Equations~(\ref{eqn:adv-strat},\ref{eqn:learner-strat-1}) define
simple strategies for the adversary and the learner. For these
strategies we prove our main result regarding the integer game.
\begin{theorem} \label{thm:integer-time-game}
  
    There exists a strategy for the adversary such that for any strategy
    of the learner, $$\score(T) \geq \E{\R \sim \Binom(T,1)}{\pot(T,\R)}$$

    There exists a strategy for the learner such that for any strategy
    of the adversary, $$\score(T) \leq \E{\R \sim \Binom(T,2)}{\pot(T,\R)}$$
  \end{theorem}

  $\Binom(n,s)$ is defined in Footnote~\ref{def:binomial}, and the 
  proof is given in Appendix~\ref{proof:integer-time-game}. 
  The potential 
  This proof is based on the concept of upper and lower potentials
  which is our main tool in this paper and is explained in the
  following section. 

\section{Potentials}

The definition of an integer time game specifies the 
The final potential $\pot(T,\R)$ is set in the definition of a game.
In this section we show a natural way to extend the definition to all game iterations.

\begin{definition}[potential backwards recursion] \label{def:back-recursion}
Let $T$ be the number of iterations, $\pot(T,\R)$ be the final
potential, $\l$ be a learner strategy and $\a$ be an
adversarial strategy. We define the {\em intermediate potential
  functions} of  $i=T-1,T-2,\ldots,0$ using the following
backwards recursion:~\footnote{To bootstrap the recursion we set
  $\potLA(T,\R) = \pot(T,\R)$}
\begin{equation}
  \potLA(i,\R) = \E{l \sim \adversM(i,\R)}{\potLA(i+1,\R+l-\ell_\l(i))}
\end{equation}
Where $\ell_\l(i)$ is defined in Eqn~(\ref{eqn:agg-loss-complex}).
\end{definition}
The following lemma guarantees that, for this definition of the
potential function, the score function does not change
from iteration to iteration.
\begin{lemma} \label{lemma:constant-score}
  Define the score at iteration $i$ to be 
  \begin{equation} \label{eqn:potential}
  \scoreLA(i) = \E{\R \sim \state(i)}{\potLA \left(i+1,\R \right)}
\end{equation}
then
$$\potLA(0,0) = \scoreLA(0) = \scoreLA(1) = \cdots = \scoreLA(T)$$
\end{lemma}
The proof is given in Appendix~\ref{proof:constant-score}

Definition~\ref{def:back-recursion} and
Lemma~\ref{lemma:constant-score} correspond to a fixed pair of
strategies. Using those, there is a natural way to define an {\em
  upper potential} $\upperpot_\l$ we characterizes the least upper
bound on the potential that is guaranteed by the learner strategy
$\l$.
\begin{equation} \label{eqn:recursive-upper}
  \upperpot_\l(i,\R) =
  \begin{cases}
    \pot(T,\R) & \mbox{ if } i=T \\
    \sup_\a \E{l \sim \a(i,\R)}{\upperpot_\l(i+1,\R+l-\ell_\l(i))} & \mbox{ if }
    0\leq i < T
  \end{cases}
\end{equation}
 Similarly, the lower potential $\lowerpot_\a$ charaecterizes
the highest lower bound on the potentail guaranteed bby the
adversarial strategy $\a$.
\begin{equation} \label{eqn:recursive-lower}
  \lowerpot_\a(i,\R) =
  \begin{cases}
    \pot(T,\R) & \mbox{ if } i=T \\
    \inf_\l \E{l \sim \a(i,\R)}{\lowerpot_\a(i+1,\R+l-\ell_\l(i))} & \mbox{ if }
    0\leq i < T
  \end{cases}
\end{equation}

\subsection{Strategies  for the integer time game}
We assign the adversary the {\em even split} strategy, described in the
introduction. Formally, even split is defined as 
\begin{equation} \label{eqn:adv-strat}
  \evensplit(i,\R) =
  \begin{cases}
    -1 & \mbox{ w.p. } 1/2\\
    +1 & \mbox{ w.p. } 1/2
  \end{cases}
\end{equation}
It is easy to see that, when $\evensplit$ is used, the expected loss
$\ell_\l=0$ regardless of $\l$. In other words, the learner has no
influence on the lower potential which is simply:
 \begin{equation} \label{eqn:value-iteration-lower}
   \lowerpoteven(i-1, \R) = \frac{\lowerpoteven(i,\R+1) + \lowerpoteven(i,\R-1)}{2}
 \end{equation}
   
 The learner strategy:
 \begin{equation} \label{eqn:learner-strat-1}
   \diffweight(i-1,\R) = \frac{1}{Z} \frac{\pot(i,\R+2) - \pot(i,\R-2)}{2}
 \end{equation}
 Where $Z$ is a normalization factor
 $$Z = \E{\R \sim \state(i)}{\frac{\pot(i,\R+2) - \pot(i,\R-2)}{2}}$$
 guarantees the upper potential 
 \begin{equation} \label{eqn:value-iteration-upper-recursion}
   \upperpotdiff(i-1, \R) = \frac{\upperpotdiff(i,\R+2) + \upperpotdiff(i,\R-2)}{2}
 \end{equation}
 
These strategies satisfy Theorem~\ref{thm:integer-time-game}, the
proof is given in Appendix~\ref{proof:integer-time-game}.
 
  We find that the lower bound corresponds to an unbiased random
  walk with step size $\pm 1$. The upper bound also corresponds to a an
  unbiased random walk with step size $\pm(1+c)$. The natural setting
  in the natural game is $c=1$, which means that there is a
  significant difference between the upper and lower bounds. As we
  show in the next section, this gap converges to zero in the
  continuous time setting, and the upper and lower bounds match,
  making the strategies for both sides min-max optimal. 

  Note also that the adversarial strategy the aggregate loss
  $\ell(t)$ is always zero, regardless of the strategy of the
  learner, and state progression is independent of the
  learner's choices.

%\subsection{Results for the discrete time game}

In the discrete time game the adversary has an additional choice, the
choice of $s_i$. Thus the adversary's strategy includes that choice.
There are two constraints on this choice: $s_i \geq 0$ and
$\sum_{i=1}^n s_i^2 = T$. Note that even that by setting $s_i$
arbitrarily small, the adversary can make the number of steps - $n$ -
arbitrarily large. We will therefor not identify a single adversarial
strategy but instead consider the supremum over an infinite sequence
of strategies.

We use $N(0,\sigma)$ to denote the normal distribution with mean 0 and
std $\sigma$.
\iffalse
We start with motivation for using time that is indexed by real
values rather than the natural numbers. We distinguish between two
notions of time. The first notion of time is a counter that
counts the iterations of the game, we will call this counter the {\em
  iteration counter } and denote it by $i=0,1,\ldots$. The second,
more interesting notion of time is the time that appears in the regret
bounds, we denote this time by $t_i$ where $i$ is the iteration
number. We restrict the time increments $\Delta t_i =t_i-t_{i-1}$ to
the range $0\leq \Delta t_i \leq 1$.  The magnitude of $\deltat_i$
corresponds to the {\em hardness} of iteration $i$. $\deltat_i=0$
corresponds to the case where the losses of all of the experts are
equal to a common value $-1 \leq a \leq 1$. In this case the aggregate
loss $\ell=a$, the state does not change:
$\state(i-1)=\state(i)$, $\Delta t_i=0$ and the instantaneous regret is zero. On the
other hand $\Delta t_i=1$ corresponds to the adversarial strategy
$\adversM^{1/2}(t-1,R)$ (Eqn.~\ref{eqn:adv-strat}) which maximizes the
instantaneous regret.

We introduce an additional step to the integer game. Before the
learner makes it's choice, the 
adversary chooses a real number $0 \leq s_i \leq 1$, by doing so, the
adversary commits that all of the instantaneous expert losses at that
step be in the range $[-s_i,s_i]$. The time step is defined to be $\deltat_i =
s_i^2$.

First, note that the adversary in this game is at least as powerful as
the adversary in the integer game. This is because the adversary can
always choose $s_i=1$, effectively reducing the game to the integer
game.

Next, we justify the choice $\deltat_i = s_i^2$. Our argument is that
any significantly different choice would give the game to one side or
the other.  Suppose that $s_i = 1/k$ on all $m_k$ iterations of the game. In
other words, this is a re-scaling of the integer game. Consider the
adversarial strategy. The distribution (state) after $m_k$ iterations is the
binomial distribution. with mean zero and variance $m_k \frac{1}{k^2}$
if $m_k \gg \frac{1}{k^2}$ then the variance at the end of the game
goes to infinity.

\iffalse
The state after $m_k$ iterations is the
binomial dist
\begin{equation}
  \score^{m(k)} = \frac{1}{2^m} \sum_{i=-m}^m {2m+1 \choose i+m} \pot(m,i/k)
\end{equation}
if $m/\deltat_i $
\fi

By allowing $\Delta t_i$ to vary from iteration to iteration
we get a more refined quantification of the regret and, as we show
below, min/max optimality.

To find the relationship between loss magnitude and time increments 
we compare two adversarial strategies.  The first strategy, discussed above,
generates losses $\pm 1$ with equal probability, we denote this
strategy by $\adversM^{1/2}_{\pm 1}$. The other strategy, denoted
$\adversM^{1/2}_{\pm 1/k}$, generates losses of $\pm 1/k$ with equal
probabilities.

From the adversarial point of view $\adversM^{1/2}_{\pm 1/k}$ is worse
than $\adversM^{1/2}_{\pm 1}$. So it should correspond to a smaller
time increment. But how much smaller? Suppose we start with the
initial state $\state(0)$ which is a delta functions at $\R=0$.  One
iteration of $\adversM^{1/2}_{\pm 1}$ results in a distribution
$\pm 1$ w.p, $(1/2,1/2)$, which has mean $0$ and variance $1$.
Suppose we associate $\Delta t =1/j$ with a single step of
$\adversM^{1/2}_{\pm 1/k}$.  Equivalently, we associate $j$ iterations
of $\adversM^{1/2}_{\pm 1/k}$ with $t=1$.  How should we set $j$? the
distribution generated by $j$ steps is a binomial distribution
supported on $j+1$ points, so there is no hope of making the two
distributions identical. However, as it turns out, it is enough to
equalize the mean and the variance of the two distributions. The mean
of $\adversM^{1/2}_{\pm 1/k}$ is zero for any $k$. As for the
variances, a single iteration of $\adversM^{1/2}_{\pm 1}$ is 1 and a
single iteration of $\adversM^{1/2}_{\pm 1/k}$ is $1/k^2$. It follows
that the variance after $j$ iterations of $\adversM^{1/2}_{\pm 1/k}$
$j/k^2$. Equating this variance with that of a single step of
$\adversM^{1/2}_{\pm 1}$ we get $j=k^2$ and $\Delta t= 1/k^2$.

Note a curious behavior of the {\em range} of $\R$ as $k \to \infty$
the number of steps increases like $k^2$ while the size of each step
is $1/k$. This means that the range of $\R$ is $[-k,k]$, which becomes
converges to $(-\infty, + \infty)$ when $k \to \infty$. On the other
hand, the variance increases like $t$.

Next lets consider effect of reducing the step size on a {\em biased}
strategy $\adversM^{1/2+\gamma}_{\pm 1}$ as defined in
Eqn~(\ref{eqn:adv-strat-p}) for some
$0\leq \gamma \leq 1/2$.  We now figure out what  $\gamma'$ should be
so that the distribution generated by $k^2$ iterations of $\adversM^{1/2+\gamma'}_{\pm 1/k}$ has the
same mean as a single iteration of $\adversM^{1/2+\gamma}_{\pm
  1}$. The mean of a single iteration of
$\adversM^{1/2+\gamma}_{\pm 1}$ is $2\gamma$ while the mean of a
single iteration of $\adversM^{1/2+\gamma'}_{\pm 1/k}$ is
$2\gamma'/k$. Therefor to keep the means equal we need to set
$2\gamma'/k = 2\gamma$ or $\gamma' = \gamma/k$.


Note that as $k \to \infty$, $\gamma' \to 0$. This observation
motivates scaling the bound on $\ell(t)$ like $c s_i^2$ (see the
description of the game below.)
\fi

\begin{theorem}
  ~\\

   let $A = \E{\R \sim N(0,\sqrt{T})}{\pot(T,\R)}$
   \begin{itemize}
     \item
    For any $\epsilon>0$ there exists a strategy for the adversary
    such that for any strategy of the learner $\score(T) \geq A-\epsilon$
  \item
    There exists a strategy for the learner that guarantees, against
    any adversary $\score(T) \leq A$.
  \end{itemize}
\end{theorem}


\subsection{The adversary prefers smaller steps} \label{sec:smallsteps}
As noted before, if the adversary chooses $s_i=1$ for all $i$ the game
reduces the the integer time game. The question is whether the
adversary would prefer to stick with $s_i=1$ or instead prefer to use
$s_i<1$. In this section we give a surprising answer to this question
-- the adversary always prefers a smaller value of $s_i$ to a larger
one. This leads to a preference for $s_i \to 0$, as it turns out, this
limit is well defined and corresponds to Brownian motion, also known as
Wiener process.

Consider a sequence of adversarial strategies $S_k$ indexed by
$k=0,1,2,$. The adversarial strategy $S_k$ is corresponds to always
choosing $s_i = 2^{-k}$, and repeating  $\adversM^{1/2}_{\pm 2^{-k}}$ 
for $T 2^{2k}$ iterations.
This corresponds to the distribution created by a random walk with
$T 2^{2k}$ time steps, each step equal to $+2^{-k}$ or  $-2^{-k}$ with probabilities $1/2,1/2$.
Note that in order to preserve the variance, halving the step size
requires increasing the number of iterations by a factor of four.

Let $\pot(S_k,t,\R)$ be the value associated with adversarial
strategy $S_k$, time $t$ (divisible by $2^{-2k}$) and
location $\R$. We are ready to state our main theorem.

\begin{theorem}\label{thm:smallerSteps}
  If the final value function has a strictly positive fourth
  derivative:
  $$ \frac{d^4}{d \R^4} \finalPot >0, \forall \R$$
  then for any integer $k>0$ and any $0 \leq  t \leq T$, such that $t$
  is divisible by
  $2^{-2k}$ and any $\R$,
  $$\pot(S_{k+1},t,\R)) >  \pot(S_{k},t,\R)$$
\end{theorem}
Proofs for Theorem~\ref{thm:smallerSteps} and
Lemma~\ref{lemma:n-strictly-convex} are given in Appendix~\ref{proofs:n-strictly-convex-smallerSteps}
Before proving the theorem, we describe it's
consequence for the online learning problem.
We can restrict Theorem~\ref{thm:smallerSteps} for the
case $t=0$,$\R=0$ in which case we get an increasing sequence:
\[
\pot(S_1,0,0) < \pot(S_2,0,0) <\cdots <\pot(S_k,0,0) <
\]
The limit of the strategies $S_k$ as $k \to \infty$ is the well
studied Brownian or Wiener process. The backwards recursion that
defines the value function is the celebrated Backwards Kolmogorov
Equation with zero drift and unit variance
\begin{equation} \label{eqn:Kolmogorov}
  \frac{\partial}{\partial t} \pot(t,\R)
  + \frac{1}{2} \frac{\partial^2}{\partial \R^2} \pot(t,\R)=0
\end{equation}
Given a final value function with a strictly positive fourth
derivative we can use Equation~(\ref{eqn:Kolmogorov}) to compute the
value function for all $0 \leq t \leq T$. We will do so in he next section.

We now go back to proving Theorem~\ref{thm:smallerSteps}. The core of
the proof is a lemma which compares, essentially, the value recursion
when taking one step of size 1 to four steps of size 1/2.


Consider the adversarial strategies $S_k$ and $S_{k+1}$ at a particular
time point $0 \leq t \leq T$ such that $t$ is divisible by
$\deltat=2^{-2k}$ and at a particular location $\R$. Let
$t'=t+\deltat$, and fix a value
function for time , $\pot(t',\R)$ and compare between
two values at $\R,t$. The first value denoted
$\pot_k(t,\R)$ corresponds to $S_k$, and consists of a single random step of $\pm 2^{-k}$. 
The other value $\pot_{k+1}(t,\R)$ corresponds to $S_{k+1}$ and consists of
four random steps of size $\pm 1/2$.

\begin{lemma} \label{lemma:n-strictly-convex}
If $\pot(t',\R)$ is, as a function of $\R$ continuous, strictly
convex and with a strictly positive fourth derivative. Then
\begin{itemize}
\item $\pot_k(t,\R) < \pot_{k+1}(t,\R)$
  \item Both $\pot_k(t,\R)$ and $\pot_{k+1}(t,\R)$ are continuous, strictly
convex and with a strictly positive fourth derivative.
\end{itemize}
\end{lemma}

\subsection{Strategies for the Learner in the discrete time game}
The strategies we propose for the learner in the continuous time game
are an adaptation of the strategies $\learnerM^1,\learnerM^2$ to the
case where $s_i<1$.

We start with the high-level idea. Consider iteration $i$ of the
continuous time game. We know that the adversary prefers $s_i$ to be
as small as possible. On the other hand, the adversary has to choose
some $s_i>0$. This means that the adversary always plays
sub-optimally. Based on $s_i$ the learner makes a choice and the
adversary makes a choice. As a result the current state $\state(t_{i-1})$
is transformed to $\state(t_i)$. To choose it's strategy, the learner
needs to assign value possible states $\state(t_i)$. How can she do
that? By assuming that in the future the adversary will play
optimally, i.e. setting $s_i$ arbitrarily small. While the adversary
cannot be optimal, it can get arbitrarily close to optimal, which is
Brownian motion.

Solving the backwards Kolmogorov equation with the boundary condition
$\pot(T,\R)$ yields $\pot(t,\R)$ for any
$\R \in \reals$ and $t \in [0,T]$. We now explain how using this
potential function we derive strategies for the the learner. 

Note that the learner chooses a distribution {\em after} the adversary
set the value of $s_i$. The discrete time version of $\learnerM^1$
(Eqn~\ref{eqn:learner-strat-1}) is 
\begin{eqnarray} \label{eqn:learner-strat-1c}
  \learnerM^{1d}(t_{i-1},\R) = \frac{1}{Z^{1d}}
  \frac{\pot(t_i,\R+s_{i-1}+cs_{i-1}^2) -
  \pot(t_i,\R-s_{i-1}-cs_{i-1}^2)}{2} \\
  \mbox{ where } Z^{1d} = \E{\R \sim \state(t_i)}{\frac{\pot(t_i,\R+s_{i-1}+cs_{i-1}^2) -
  \pot(t_i,\R-s_{i-1}-cs_{i-1}^2)}{2}} \nonumber
\end{eqnarray}

Next, we consider the discrete time version of $\learnerM^2$:
(Eqn~\ref{eqn:learner-strat-2})
\begin{eqnarray} \label{eqn:learner-strat-2c}
  \learnerM^{2d}(t_{i-1},\R) =  \frac{1}{Z^{2d}}
  \left. \frac{\partial}{\partial r} \right|_{r=\R}
  \pot(t_{i-1}+s_{i-1}^2,r)
  \\
  \mbox{ where } Z^{2d} = \E{\R \sim
  \state(t_i)}{\left. \frac{\partial}{\partial r} \right|_{r=\R} \pot(t_{i-1}+s_{i-1}^2,r)} \nonumber
\end{eqnarray}


\section{Two KBE compatible potential functions} \label{sec:self-consistent}

The potential functions, $\pot(t,\R)$ is a solution of PDE~(\ref
{eqn:Kolmogorov}):
\begin{equation} 
  \frac{\partial}{\partial t} \pot(t,\R)
  + \frac{1}{2} \frac{\partial^2}{\partial r^2} \pot(t,\R)=0
\end{equation}
under a boundary condition $\pot(T,\R)=\finalPot$, which we assume
is in $\SP{4}$

So far, we assumed that the game horizon $T$ is known in advance. We
now show two value functions where knowledge of the horizon is not
required. Specifically, we call a value function $\pot(t,\R)$
{\em self consistent} if it is defined for all $t>0$ and if for any
$0<t<T$, setting $\phi(T,\R)$ as the final potential and solving for
the Kolmogorov Backward Equation yields $\phi(t,\R)$ regardless of the
time horizon $T$. 

We consider two solutions to the PDE, the exponential potential and
the NormalHedge potential. We give the form of the potential function
that satisfies Kolmogorov Equation~\ref{eqn:Kolmogorov}, and derive
the regret bound corresponding to it.

{\bf The exponential potential function} which corresponds to exponential
  weights algorithm corresponds to the following equation
\[
    \pot_{\mbox{exp}}(\R,t) = e^{\sqrt{2} \eta \R - \eta^2 t}
  \]
  Where $\eta>0$ is the learning rate parameter.
  
Given $\epsilon$ we choose $\eta = \sqrt{\frac{\ln (1/\epsilon)}{t}}$
we get the regret bound that holds for any $t>0$
  \begin{equation}
    \R_\epsilon \leq \sqrt{2 t \ln \frac{1}{\epsilon}}
  \end{equation}
Note that the algorithm depends on the choice of $\epsilon$, in other
words, the bound does {\em not} hold for all values of $\epsilon$ at
the same time.

{\bf The NormalHedge value} is
\begin{equation} \label{eqn:NormalHedge}
  \pot_{\mbox{NH}}(\R,t) = \begin{cases}
    \frac{1}{\sqrt{t+\nu}}\exp\left(\frac{\R^2}{2(t+\nu)}\right)
    & \mbox{if } \R \geq 0  \\
  \frac{1}{\sqrt{t+\nu}} & \mbox{if } \R <0
  \end{cases}
\end{equation}
Where $\nu>0$ is a small constant. The function $\pot_{\mbox{NH}}(\R,t)$,
restricted to $\R\geq 0$ is in $\SP{4}$ and is a constant for $\R \leq 0$.

The regret bound we get is:
\begin{equation}
\R_\epsilon \leq \sqrt{(t+\nu) \left( \ln (t+\nu) + 2 \ln \frac{1}{\epsilon}\right)}
\end{equation}
This bound is slightly larger than the bound for exponential weights,
however, the NormalHedge bound holds simultaneously for all
$\epsilon>0$ and the algorithm requires no tuning.


\section{NormalHedge yields the fastest increasing potential} \label{sec:NormalHedge}

Up to this point, we considered any continuous value function with
strictly positive derivatives 1-4. We characterized the min-max
strategies for any such function. It is time to ask whether value
functions can be compared and whether there is a ``best'' value
function. In this section we give an informal argument that
NormalHedge is the best function. We hope this argument can be
formalized.

We make two observations. First, the min-max strategy for the
adversary does not depend on the potential function! (as long as it
has strictly positive derivatives). That strategy corresponds to the
Brownian process.

Second, the argument used to show that the regret relative to
$\epsilon$-expert of the expert is based on two arguments
\begin{itemize}
\item The average value function does not increase with time.
\item The (final) value function increases rapidly as a function of $\R$
\end{itemize}
The first item is true by construction. The second argument suggests
the following partial order on value functions. Let
$\pot_1(t,\R),\pot_2(t,\R)$ be two value functions such that
\[
\lim_{\R \to \infty} \frac{\pot_1(t,\R)}{\pot_2(t,\R)} = \infty  
\]
then $\pot_1$ {\em dominates} $\pot_2$, which we denote by, $\pot_1 > \pot_2$.

On the other hand, if the value function increases too quickly, then,
when playing against Brownian motion, the average value will increase
without bound.  Recall that the distribution of the Brownian process
at time $t$ is the standard normal with mean 0 and variance $t$.
The question becomes what is the fastest the value
function can grow, as a function of $\R$ and still have a finite
expected value with respect to the normal distribution.

The answer seems to be NormalHedge (Eqn.~\ref{eqn:NormalHedge}). More
precisely, if $\epsilon>0$, the mean value is finite, but if
$\epsilon=0$ the mean value becomes infinite.

\bibstyle{plain}
\bibliography{ref,bib}
\appendix


\section{Proof of Theorem~\ref{thm:integer-time-game}}
\label{proof:integer-time-game}
\begin{proof}
\begin{enumerate}
\item By symmetry adversarial strategy~(\ref{eqn:adv-strat}) guarantees that
  the aggregate loss~(\ref{eqn:aggregate-loss}) is zero regardless of
  the choice of the learner: $\ell(t)=0$.
  Therefor the state update~(\ref{eqn:state-update}) is equivalent to
  the symmetric random walk:
  $$\state(i) = \frac{1}{2} \paren{(\state(i-1) \oplus 1) + (\state(i-1)
    \ominus 1)}$$
  Which in turn implies that if the adversary plays $\adversM^*$
  and the learner plays an arbitrary strategy $\learnerM$
  \begin{equation} \label{eqn:lower}
    \lowerpot(i-1,\R) = \frac{1}{2} \paren{\pot(i,\R-1)+\pot(i,\R+1)}
  \end{equation}
  As this adversarial strategy is oblivious to the strategy, it
  guarantees that the average value at iteration $i$ is {\em equal} to the
  average of the lower value at iteration $i-1$.
\item
 Plugging learner's strategy~(\ref{eqn:learner-strat-1}) into equation~(\ref{eqn:aggregate-loss}) we find that
 \begin{equation} \label{eqn:ell-optimal-learner}
   \ell(i-1) = \frac{1}{Z_{i-1}} \E{\R \sim \state(i-1)}{\paren{\pot(i,\R+1+c)-\pot(i,\R-1-c)}
   \Bias(i-1,\R)}
\end{equation}
  Consider the average value at iteration $i-1$ when the learner's strategy
  is $\learnerM^*$ and the adversarial strategy is arbitrary $\adversM$:
  
   \begin{equation} \label{eqn:Pot-Update}
    \score_{\learnerM^*,\adversM}(i-1,\R) = \E{\R \sim \state(i-1)}{ \E{y \sim
      \adversM(i-1)(\R)}{\pot(i,\R+y-\ell(i-1))}}
  \end{equation}
  As $\pot(i,\cdot)$ is convex and as $(y-\ell(i-1)) \in [-1-c,1+c]$,
  \begin{equation} \label{eqn:pot-upper}
    \pot(i,\R+y) \leq \frac{\pot(i,\R+1+c)+\pot(i,\R-1-c)}{2} +
    (y-\ell_\l(i)) \frac{\pot(i,\R+1+c)-\pot(i,\R-1-c)}{2}
    \end{equation}
  Combining the equations~(\ref{eqn:ell-optimal-learner}) and~(\ref{eqn:Pot-Update}) we find that
  \begin{eqnarray}
  \score_{\learnerM^*,\adversM}(i-1,\R)&=&\E{\R \sim \state(i-1)}{\E{y \sim \adversM(i-1)(\R)}{\pot(i,\R+y-\ell(i-1))}}\\
  &\leq & \E{\R \sim \state(i-1)}{\frac{\pot(i,\R+1+c)+\pot(i,\R-1-c)}{2}}\\
  &+&
  \E{\R \sim \state(i-1)}{\E{y \sim \adversM(i-1)(\R)}{(y-\ell(i-1)) \frac{\pot(i,\R+1+c)-\pot(i,\R-1-c)}{2}}} \label{eqn:zero-term}
  \end{eqnarray}
  
The final step is to show that the term~(\ref{eqn:zero-term}) is equal
to zero. As $\ell(i-1)$ is a constant with respect to $\R$ and $y$ the
term~(\ref{eqn:zero-term}) can be written as:
\begin{eqnarray}
&&\E{\R \sim \state(i-1)}{\E{y \sim \adversM(i-1)(\R)}{(y-\ell(i-1))
   \frac{\pot(i+1,\R+1)-\pot(i+1,\R-1)}{2}}}\\
&=&
\E{\R \sim \state(i-1)}{\Bias(i-1,\R)
    \frac{\pot(i,\R+1+c)-\pot(i,\R-1-c)}{2}}\\
  &-& \ell_\l(i) \E{\R \sim \state(i-1)}{
    \frac{\pot(i,\R+1+c)-\pot(i,\R-1-c)}{2}}\\
  &=& 0
\end{eqnarray}
\end{enumerate}
\end{proof}

\section{Proof of Lemma~\ref{lemma:constant-score}}
\label{proof:constant-score}
\begin{proof}
We prove the lemma by showing that  $\scoreLA(i) = \scoreLA(i+1)$ for
all $i \in \{T-1,T-2,\ldots,0\}$

\begin{eqnarray}
  \scoreLA(i+1) &=& \E{\R \sim \state(i+1)}{\potLA \left(i+1,\R \right)}\\
                &=& \E{\R \sim \state(i),\; l \sim
                    \adversM(i,R)}{\potLA \left(i+1,\R+l-\ell_\l(i)
                    \right)}\\
                &=& \E{\R \sim \state(i)}{\E{l \sim
                    \adversM(i,R)}{\potLA \left(i+1,\R+l-\ell_\l(i)
                    \right)}}\\
                &=& \E{\R \sim \state(i)}{\potLA(i,\R)} \\
                &=& \scoreLA(i)
\end{eqnarray}
\end{proof}

\section{Proofs of Lemma~\ref{lemma:n-strictly-convex} and Theorem~\ref{thm:smallerSteps}}
\label{proofs:n-strictly-convex-smallerSteps}
\begin{proof}{\bf of Lemma~\ref{lemma:n-strictly-convex}}
Recall the notations $\deltat = 2^{-2k}$ $t' = t+\deltat$ and $s=2^{-k}$.
We can write out explicit expressions for the two values:
\begin{itemize}
\item For strategy $S_0$ the value is
  $$\pot_k(t, \R) = \frac{\pot(t',\R+s)+ \pot(t',\R-s)}{2} $$.
\item For strategy $S_1$ the value is
  $$\pot_{k+1}(t, \R) = \frac{1}{16}
  \paren{\pot(t',\R+2s)+ 4\pot(t',\R+s)+ 6\pot(t',\R)+  4\pot(t',\R-s)+ \pot(t',\R-2s)}
  $$.
\end{itemize}

We want to show that $\pot_1(T-1,\R) > \pot_0(T-1,\R)$ for all $\R$, in
other words we want to characterize the properties of $\finalPot$ the
would guarantee that
\begin{equation} \label{eqn:4thOrderConvex}
\pot_1(t,\R) - \pot_0(t,\R) =
\frac{1}{16}
\paren{ \pot(t',\R+2) - 4\pot(t',\R+1) +6\pot(t',\R) - 4\pot(t',\R-1) +\pot(t',\R-2)} > 0
\end{equation}

Inequalities of this form have been studied extensively under the name
``divided differences''~\cite{popoviciu1965certaines,butt2016generalization, de2005divided}.
A function $\finalPot$ that satisfies
inequality~\ref{eqn:4thOrderConvex} is said to be {\em 4'th order convex}
(see details in in~\cite{butt2016generalization}).

$n$-convex functions have a very simple characterization:
\begin{theorem}
  Let $f$ be a  function with is differentiable up to order $n$, and
  let $f^{(n)}$ denote the $n$'th derivative, then $f$ is $n$-convex
  ($n$-strictly convex) if and only if $f^{(n)} \geq 0$ ($f^{(n)} > 0$).
\end{theorem}

We conclude that if $\pot(t',\R)$ has a strictly positive fourth
derivative then $\pot_{k+1}(t,\R) > \pot_{k}(t,\R)$ for all $\R$, proving
the first part of the lemma.

The second part of the lemma follows from the fact that
both $\pot_{k+1}(t,\R)$ and $\pot_{k}(t,\R)$ are convex combinations of
$\pot(t,\R)$ and therefor retain their continuity and convexity properties.
\end{proof}

\begin{proof}  of Theorem~\ref{thm:smallerSteps} \\
The proof is by double induction over $k$ and over $t$.
For a fixed $k$ we take a finite backward induction over
$t=T-2^{-2k},T-2 \times 2^{-2k},T-3 \times 2^{-2k},\cdots,0$.
Our inductive claims are that $\pot_{k+1}(t,\R) > \pot_{k}(t,\R)$ and
$\pot_{k+1}(t,\R)$,$\pot_{k}(t,\R)$ are continuous, strongly convex and
have a strongly positive fourth derivative. That these claims carry over
from $t=T-i \times 2^{-2k}$ to  $t=T-(i+1) \times 2^{-2k}$ follows
directly from Lemma~\ref{lemma:n-strictly-convex}.

The theorem follows by forward induction on $k$.

\end{proof}

\end{document}



\iffalse
a generalization of the of online prediction with expert
advice called {\em the decision-theoretic online learning game}
(DTOL)~\cite{freund1997decision}, we  consider the {\em
  signed} version of the online game.

DTOL is a repeated zero sum game between a {\em learner} and an {\em
  adversary}. The adversary controls the losses of $N$ experts, or
(also referred to as actions), while the learner controls a
distribution over the experts.

Iteration $i=1,\ldots,T$ of the game consists of the following steps:
\begin{enumerate}
    \item The learner chooses a distribution $P_j^i$ over the
      experts $j \in
      \{1,\ldots,N\}$. 
    \item The adversary chooses an {\em instantaneous loss} for each
      of the $N$ experts: $l_j^i \in [-1,+1]$
      for $j \in \{1,\ldots,N\}$.
    \item The learner incurs an {\em expected loss} defined to be
      $P_i$: $\ell^i = \sum_{j=1}^N P_j^i l_j^i$
\end{enumerate}

A {\em strategy} for the learner (adversary) is a mapping from the
history of choices made by both sides on previous iterations to the
choice that is made on the current iteration.
    
We denote the {\em cumulative loss} of expert $j$ on iteration $k$ by $L^k_j =
\sum_{i=1}^k l_j^i$.
Similarly, we denote the {\em cumulative loss} of the learner by 
$L_L^k = \sum_{i=1}^k \ell^i$.
The {\em instantaneous regret} of the learner relative to expert
$j$ on iteration $i$ is defined as $r_j^i = \ell^i - l_j^i$. 
The {\em cumulative regret} of the learner with respect to expert
$j$ for iterations $i=1,\ldots, k$ is $\R_j^k = \sum_{i=1}^k r_j^i$.
We denote the vector of all $N$ regrets at iteration $k$ by $\vR^k=\{\R_j^k\}_{j=0}^N$

The goal of the learner is to minimize the maximal regret at the end
of the game $R_*^T \doteq \max_j \R_j^T = L_L^T - \min_j L_j^T$.
The goal of the adversary is to maximize the same quantity. We use 
$L_*^T = \min_j L_j^T$ to denote the total loss of the best
expert, and write $R_*^T =  L_L^T - L_*^T$. This paper
describes strategies for the learner that provide upper bounds on
$R_*^T$ and strategies for the adversary that provide lower bounds on $R_*^T$.

\subsection{Some known Bounds}

Zero-order bounds on the regret ~\cite{freund1999adaptive} depend only on $N$
and $T$ and typically have the form
\begin{equation} \label{eqn:0-order-bound}
  \max_j \R_j^T < C \sqrt{T \ln N}
\end{equation}
for some small constant $C$ (typically smaller than 2).
These bounds can be extended to infinite sets of experts by defining
the $\epsilon$-regret of the algorithm as the regret with respect to
the best (smallest cumulative loss) $\epsilon$-percentile of the set of experts.
this replaces the bound~(\ref{eqn:0-order-bound}) with 
\begin{equation} \label{eqn:0-epsilon-order-bound}
  \max_j \R_j^T < C \sqrt{T \ln \frac{1}{\epsilon}}
\end{equation}



Lower bounds have been proven that match these upper bounds up to a
constant. These lower bounds typically rely on constructions in which
the losses $l_j^i$ are chosen independently at random to be either
$+1$ or $-1$ with equal probabilities.

\fi

\iffalse
\section{Introduction} 
Online prediction with expert advise has been studied extensively over
the years and the number of publications in the area is vast. For some
starting points see~\cite{vovk1990aggregating, feder1992universal,
  littlestone1994weighted, cesa1997use, cesa2006prediction}.

In this paper we improve on the analysis of the normal-hedge
algorithm~\cite{chaudhuri2009parameter, luo2015achieving}. In
particular provide an algorithm, parametrized by $\nu>0$
achieves a regret bounds of the form
\begin{equation} \label{eqn:bound}
\R_\epsilon \leq \sqrt{(T+\nu) \left( \ln (T+\nu) + 2 \ln \frac{1}{\epsilon}\right)}
\end{equation}
Where $\R_{\epsilon}$ is the regret relative to the top
$\epsilon$-percentile. This bound holds for all values of $\epsilon>0$
and $T$ {\em simultanously}. The parameter $\nu>0$ has
to be chosen a-priori, setting $\nu=1$ is reasonable if $T$ is large.

This bound matches the bound of $\sqrt{t(1+\log 1/\epsilon)}$ proven
in \cite{chernov2010prediction,orabona2016coin}, however, these bounds
hold for pre-defined $\epsilon$ and $T$ which determine the learning rate.


Regret bounds of the form $\sqrt{T \ln {1/\epsilon}}$ have lower
bounds that match up to a constant. However, this assumption is too
essimistic in many real-world settings. There is a line of work on
online algorithms that perform better than this bound for ``easy
data''. An incomplete list include
\cite{CesabianchiFrHeHaScWa97,cesa2007improved,hazan2010extracting,erven2011adaptive,gaillard2014second}

Of those, particularly relevant here is the second-order
regret bound given in theorem 5 of ~\cite{cesa2007improved}
The bound given in Theorem~5 of ~\cite{cesa2007improved}:
\begin{equation} \label{eqn:2nd-order-bound}
  \max_j \R_j^T \leq 4\sqrt{V_T \ln N} +2 \ln N +1/2 
\end{equation}
Where
\[
  \var_i = \sum_{j=1}^N P^i_j (l_j^i)^2 -  \left( \sum_{j=1}^N P^i_j
    l_j^i \right)^2 \mbox{ and } \V_T= \sum_{i=1}^T \var_i
\] 
Like other algorithms based on the exponential weights, achiecing this
bound requires a-priori knowledge of $V_T$. We will show that
normal-hedge achieves the bound~\ref{eqn:bound} where $T$ is replaced
by $V_T$.
\fi
