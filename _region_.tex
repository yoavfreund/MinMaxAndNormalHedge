\message{ !name(BrownianIsWorstCase.tex)}\documentclass{article}[12pt]
\usepackage{fullpage}
% Packages
\usepackage{amsfonts}
\usepackage{amsmath}
\usepackage{amsthm}
\usepackage{graphicx}

%\theoremstyle{plain}
\usepackage{amsmath,bm}

\newtheorem{lemma}{Lemma}
\newtheorem{claim}[lemma]{Claim}
\newtheorem{theorem}[lemma]{Theorem}
\newtheorem{corollary}[lemma]{Corollary}
\newtheorem{definition}{Definition}
\newtheorem{question}{Question}

\newcommand{\note}[1]{\par{\bf Note:}#1\par}
\newcommand{\notem}[1]{{\marginpar{\tiny #1}}}

\newcommand{\figline}{\rule{\textwidth}{1pt}}

%\newcommand{\proof}{\noindent{\bf Proof:} }
%\newcommand{\qed}{\rule{0.7em}{0.7em}}

\newcommand{\newmcommand}[2]{\newcommand{#1}{{\ifmmode {#2}\else\mbox{${#2}$}\fi}}}
\newcommand{\newmcommandi}[2]{\newcommand{#1}[1]{{\ifmmode {#2}\else\mbox{${#2}$}\fi}}}
\newcommand{\newmcommandii}[2]{\newcommand{#1}[2]{{\ifmmode {#2}\else\mbox{${#2}$}\fi}}}
\newcommand{\newmcommandiii}[2]{\newcommand{#1}[3]{{\ifmmode {#2}\else\mbox{${#2}$}\fi}}}

\newcommand{\E}[2]{{\bf E}_{#1}\left[ #2 \right]}
%\newcommand{\D}[2]{{\bf D}_{#1}\left[ #2 \right]}
\newcommand{\D}{{\cal D}}

\renewcommand{\P}[2]{{\bf P}_{#1}\left[ #2 \right]}
\newcommand{\reals}{\mathbb{R}}

\newcommand{\argmin}{\mbox{argmin}}
\newcommand{\argmax}{\mbox{argmax}}
\newcommand{\SP}[1]{{\cal P}^{#1}} % strictly positive function of
                                   % order i
\newcommand{\R}{R}      % Cumulative regret
\newcommand{\vR}{\mathbf{R}} %regret vector

\renewcommand{\r}{r}      % Instantaneuous regret
\newcommand{\state}{{\bf \Psi}}
\newcommand{\1}[1]{{\mathbf 1}\left[#1\right]} % point mass at #1

\newcommand{\pot}{\phi}
\newcommand{\potPQ}{\pot_{\learnerM,\adversM}}
\newcommand{\potLA}{\pot_{\l,\a}}
\newcommand{\finalPot}[1]{\pmb{\phi}_{#1}}
\newcommand{\finalPotT}{\finalPot{T}}
\newcommand{\finalPotR}{\finalPot{\realT}}
\newcommand{\upperpot}{\pot_{\learnerM}^{\downarrow}}
\newcommand{\upperpotb}{\pot_{\learnerMb}^{\downarrow}}
\newcommand{\upperpotd}{\pot_{\learnerMd}^{\downarrow}}
%\newcommand{\upperpotj}{\pot_{\learnerM(j)}^{\downarrow}}
\newcommand{\upperpotMdk}{\pot_{\learnerMdk}^{\downarrow}}
\newcommand{\upperpotMdj}{\pot_{\learnerMdj}^{\downarrow}}


\newcommand{\lowerpot}{\pot_{\adversM}^{\uparrow}}
\newcommand{\lowerpotb}{\pot_{\adversMb}^{\uparrow}}
\newcommand{\lowerpotd}{\pot_{\adversMd}^{\uparrow}}
\newcommand{\lowerpotj}{\pot_{\adversM(j)}^{\uparrow}}
\newcommand{\lowerpotMdk}{\pot_{\adversMdk}^{\uparrow}}
\newcommand{\lowerpoteven}{\lowerpot_{\evensplit}}

\newcommand{\realT}{\mathcal{T}}  % the final time for the real-time
% game
\newcommand{\Tset}[1]{\pmb{T}_{#1}}
\newcommand{\Ilat}[1]{\pmb{I}_{#1}}
\newcommand{\Klat}[1]{\pmb{K}_{#1}}
\newcommand{\score}{\Phi}
\newcommand{\upperscore}[1]{\score_{#1}^{\downarrow}}
\newcommand{\lowerscore}[1]{\score_{#1}^{\uparrow}}

\newcommand{\upperscoreM}{\upperscore{\learnerM}}
\newcommand{\upperscoreMd}{\upperscore{\learnerMd}}
\newcommand{\upperscoreMdk}{\upperscore{\learnerMdk}}

\newcommand{\lowerscoreM}{\lowerscore{\adversM}}
\newcommand{\lowerscoreMd}{\lowerscore{\adversMd}}
\newcommand{\learnerM}{P}
\newcommand{\learnerMb}{P_I}
\newcommand{\learnerMd}{P_D}
\renewcommand{\l}{\learnerM}
\newcommand{\learnerMdk}{P_{D(k)}}
\newcommand{\learnerMdj}{P_{D(j)}}


\newcommand{\legalLearner}{{\cal L}}

\newcommand{\adversM}{Q}
\newcommand{\adversMb}{Q_I}  %Integer time game
\newcommand{\adversMd}{Q_D}  % Discrete time game
\newcommand{\adversMdk}{Q_{D(k)}}  %Discrete time game with step of
                                %size 2^{-k}


% \renewcommand{\a}{\adversM}
%\newcommand{\legalAdversary}{{\cal A}}
\newcommand{\agloss}{v}
\newcommand{\Bias}{B}
\newcommand{\deltat}{\Delta t}

\newcommand{\diffweight}{\l^d}   % learner strategy based on taking a difference
\newcommand{\upperpotdiff}{\upperpot_{\diffweight}}


\newcommand{\Binom}{\mathbb{B}}
\newcommand{\radsum}{\Binom(s_1,\ldots,s_T)}
\newcommand{\var}{\mbox{Var}}
\newcommand{\V}{V}


\newcommand{\at}[1]{\left\{ \left. #1
    \right|_{\begin{tiny}\begin{matrix}
          \tau,\rho=\\t_i,\R \end{matrix} \end{tiny}}
    \pot(\tau,\rho)\right\}}
\newcommand{\att}[1]{\left\{ \left. #1  
\right|_{\begin{tiny}\begin{matrix} \tau,\rho=\\t_i+g \deltat_i,\R_i+g
      r_i \end{matrix} \end{tiny}}
\pot(\tau, \rho)\right\}}


\newcommand{\atI}[1]{\left\{ \left. #1  
\right|_{\begin{tiny}\begin{matrix}
      x,y=\\x_0,y_0 \end{matrix} \end{tiny}}
f(x,y) \right\}}
\newcommand{\atII}[1]{\left\{ \left. #1
\right|_{\begin{tiny}\begin{matrix}
      x,y=\\x_0+t\Dx,y_0+t\Dy \end{matrix} \end{tiny}}
f(x,y) \right\}}


\newmcommandi{\paren}{\left({#1}\right)}
\newmcommandi{\brac}{\left[{#1}\right]}




\title{Potential-based hedging algorithms}
\author{Yoav Freund}
\begin{document}

\message{ !name(BrownianIsWorstCase.tex) !offset(1113) }
\begin{equation} \label{eqn:NormalHedge}
  \pot_{\mbox{\tiny NH}}(\R,t) = \begin{cases}
    \frac{1}{\sqrt{t+\nu}}\exp\left(\frac{\R^2}{2(t+\nu)}\right)
    & \mbox{if } \R \geq 0  \\
  \frac{1}{\sqrt{t+\nu}} & \mbox{if } \R <0
  \end{cases}
\end{equation}
\message{ !name(BrownianIsWorstCase.tex) !offset(1617) }

\end{document}

%% deleted stuff


The next Lemma is the main part of the proof of
Theorem~(\ref{thm:IntegerGameBounds}). We use the backward induction
from Theorem~(\ref{thm:backward-recursion}) To compute upper and lower
potentials (Equations~(\ref{eqn:upperPotentials},\ref{eqn:lowerPotentials})) for
Strategies~(\ref{eqn:adv-strat-p}) and~(\ref{eqn:learner-strat-1})

The last iteration of the game: $i=T$ is the first step of the
backward induction. The upper and lower bounds are both set equal to
the first step in the backward induction we define
$$  \lowerpotb(T,\R) = \upperpotb(T,\R) = \pot(T,\R) $$

\iffalse
\begin{lemma} \label{lemma:first-order-bound}
  If $\pot(i,\R) \in \SP{2}$
  \begin{enumerate}
    \item The adversarial strategy~(Eq~(\ref{eqn:adv-strat-p}))
    guarantees the lower potential
 \begin{equation} \label{eqn:backward-iteration-lower}
   \lowerpotb(i, \R) = \frac{\lowerpotb(i,\R+1) + \lowerpotb(i,\R-1)}{2}
 \end{equation}
   
    \item The learner strategy~(Eq~(\ref{eqn:learner-strat-1}))
      guarantees the upper potential 
      \begin{equation} \label{eqn:backward-iteration-upper-recursion}
        \upperpotb(i, \R) = \frac{\upperpotb(i,\R+2) + \upperpotb(i,\R-2)}{2}
      \end{equation}
    \end{enumerate}
\end{lemma}
\fi


  \begin{eqnarray}
  \upperpotd(t_i,\R)&=&\E{\R \sim \state(i)}{\E{y \sim \adversM(t_i,\R)}{\upperpotd(t_i,\R+y-\ell(t_i))}}\\
  &\leq & \E{\R \sim \state(t_i)}{\frac{\upperpotd(i,\R+s_k(1+s_k))+\upperpotd(i,\R-s_k(1+s_k))}{2}}\\
  &+&
      \E{\R \sim \state(i)}{\E{y \sim \adversM(i)(\R)}{(y-\ell(i))
      \frac{\upperpotd(i,\R+s_k+s_k^2)+\upperpotd(i,\R-s_k-s_k^2)}{2}}}
  \end{eqnarray}


(Eq~\ref{eqn:learner-strat-1}). We follow the same line of argument as the second part of the proof of
Lemma~\ref{lemma:first-order-bound} to give a recursion for the upper
potential. The critical difference between the integer game is and the
discrete game is that in the discrete game $\ell(t_i)\leq s_i^2$ which
implies that $(y-\ell(t_i)) \in [-s_k(1+s_k),s_k(1+s_k)]$. This yields 


  Following the same line of argument as the first part of the proof of
Lemma~\ref{lemma:first-order-bound} we consider the time points:
$t_i=i s_k^2=i 2^{-2k}\realT$ for $i=0,1,\ldots,2^{2k}$.

  \begin{equation} \label{eqn:lower-discrete}
    \lowerpotd(t_{i-1},\R) = \frac{\lowerpotd(t_i,\R-s_k)+\lowerpotd(t_i,\R+s_k)}{2}
  \end{equation}


  %%%%%

  
Follow the proof of
  Lemma~\ref{lemma:first-order-bound}.Derive upper and lower scores
  for the two strategies. Show that they converge to the same thing as
  $k \to \infty$.




We start with the high-level idea. Consider iteration $i$ of the
continuous time game. We know that the adversary prefers $s_i$ to be
as small as possible. On the other hand, the adversary has to choose
some $s_i>0$. This means that the adversary always plays
sub-optimally. Based on $s_i$ the learner makes a choice and the
adversary makes a choice. As a result the current state $\state(t_{i})$
is transformed to $\state(t_i)$. To choose it's strategy, the learner
needs to assign value possible states $\state(t_i)$. How can she do
that? By assuming that in the future the adversary will play
optimally, i.e. setting $s_i$ arbitrarily small. While the adversary
cannot be optimal, it can get arbitrarily close to optimal, which is
Brownian motion.

Note that the learner chooses a distribution {\em after} the adversary
set the value of $s_i$. The discrete time version of $\learnerM^1$

In the discrete time game the adversary has an additional choice, the
choice of $s_i$. Thus the adversary's strategy includes that choice.
There are two constraints on this choice: $s_i \geq 0$ and
$\sum_{i=1}^n s_i^2 = T$. Note that even that by setting $s_i$
arbitrarily small, the adversary can make the number of steps - $n$ -
arbitrarily large. We will therefor not identify a single adversarial
strategy but instead consider the supremum over an infinite sequence
of strategies.

%%% Local Variables:
%%% mode: latex
%%% TeX-master: t
%%% End:
